\documentclass{beamer}
\usetheme{default}

%%%%%%%%%%%%%%%%%%%%%%%%%%%%%%%%%STANDARD%%%%%%%%%%%%%%%%%%%%%%%%%%%%%%%%%
\usepackage{amsmath} % Mathematische Ausdrücke
\usepackage{amssymb}

\usepackage{amsthm} % Satz- und Beweisumgebungen

\theoremstyle{plain}
%\newtheorem{Satz}{Satz}[section]
%\newtheorem{Lemma}[Satz]{Lemma}
%\newtheorem{Korollar}[Satz]{Korollar}

\theoremstyle{definition}
%\newtheorem{Beispiel}[Satz]{Beispiel}
%\newtheorem{Definition}[Satz]{Definition}

\theoremstyle{remark}
%\newtheorem{Bemerkung}[Satz]{Bemerkung}
%\newtheorem*{Bem.}{Bemerkung}
%\newtheorem*{Konv.}{Konvention}
%\newtheorem*{Mot.}{Motivation}
%\newtheorem*{Hist. Bem.}{Historische Bemerkung}

\usepackage[english]{babel} % Silbentrennung neue Rechtschreibung
\usepackage[utf8]{inputenc} % Umlaute
\usepackage{dsfont} % Indikatorfunktion: \mathds{1}
\newcommand{\Var}{\operatorname{Var}} % Varianz
\usepackage{ulem} % Unterstreichungen
\usepackage{wasysym} % Männlich: \mars, Weiblich: \venus
\usepackage{color} % Schriftfarbe, \color{green}
%\usepackage{pxfonts} % X unabh. Y: Y \Perp Y
\definecolor{offgreen}{rgb}{.4,.6,.2}
\definecolor{offred}{rgb}{.8,0,0} 
\def\W{\color[rgb]{1,1,1}} % OliveGreen
\usepackage{graphicx}

\setlength{\parindent}{0em} % Einrücken verhindern
\usepackage{tikz} % Abbildungen erstellen
\usetikzlibrary{trees} % Stammbaum erstellen
\beamertemplatenavigationsymbolsempty %Beamer Symbolleiste aus
\DeclareMathAlphabet{\pazocal}{OMS}{zplm}{m}{n}

\usepackage{newpxmath,newpxtext} % Pfeile tiefer beschriften
% new \oset macro
\makeatletter
\newcommand{\oset}[3][-0.4ex]{%
  \mathrel{\mathop{#3}\limits^{
    \vbox to#1{\kern-2\ex@
    \hbox{$\scriptstyle#2$}\vss}}}}
\makeatother
% X mit Rückwärtshaken
\newcommand{\Xinv}{\oset{\leftharpoonup}{X}}

\newcommand{\R}{\mathbb R}
\newcommand{\C}{\mathbb C}
\newcommand{\Q}{\mathbb Q}
\newcommand{\Rd}{{\mathbb R^d}}
\newcommand{\T}{\mathbb{T}}
\newcommand{\1}{\mathbbm{1}}
\newcommand{\E}{{\mathbb E}}
\newcommand{\Pb}{\mathbb P}
\newcommand{\N}{{\mathbb N}}
\newcommand{\A}{\mathbb A}
\newcommand{\Z}{{\mathbb Z}}
\newcommand{\m}{\text{-}}
\newcommand{\eps}{\varepsilon}
\newcommand{\ph}{\varphi}
\newcommand{\rh}{\varrho}
\newcommand{\thet}{\vartheta}
\newcommand{\eqfd}{\overset{\mathrm{f.d.}}{=\joinrel=}}
%%%%%%%%%%%%%%%%%%%%%%%%%%%%%%%%%STANDARD%%%%%%%%%%%%%%%%%%%%%%%%%%%%%%%%%


\begin{document}


%%%%%%%%%%%%%%%%%%%%%%%%%%%%%%%%%%%%%%%%%%%%%%%%%%%%%%%%%%%%%%%%%%%%%%%%%%
\begin{frame}
\thispagestyle{empty}
\begin{center}
\textbf{{\LARGE Parabolic Fractal Geometry \\[12pt] of \\ [12pt] Stable L\'{e}vy Processes with Drift}}\\[42pt]

Leonard Pleschberger\\[12pt]
Heinrich Heine University D\"{u}sseldorf

\quad\\[50pt]

\date{mylongdate}{21.}{05.}{2024}\\[12pt]

\end{center}
\end{frame}

\thispagestyle{empty}


%%%%%%%%%%%%%%%%%%%%%%%%%%%%%%%%%%%%%%%%%%%%%%%%%%%%%%%%%%%%%%%%%%%%%%%%%%
\begin{frame}{Agenda}
\begin{enumerate}
\item Introduction: $\alpha$-stable L\'{e}vy processes $X = (X_t)_{t\geq0}$ in $\Rd$\\[12pt]
\item Stochastic formulation of PDEs\\[12pt]
\item $\alpha$-parabolic Hausdorff dimension\\[12pt]
\item Formula: Hausdorff dimension of $X +f$\\[12pt]
\item Estimates for the parabolic Hausdorff dimension
\end{enumerate}
\end{frame}


\begin{frame}{Introduction: $\alpha$-Stable L\'{e}vy Processes $X = (X_t)_{t\geq0}$}
\begin{enumerate}
\item Time-continuous stochastic process where $\alpha \in (0,2]$;\\ $\alpha = 2$: Brownian Motion \\[12pt]
\item L\'{e}vy process: Independent and stationary increments, stochastically continuous, starts a.s. in $0 \in \Rd$\\[12pt]
\item Transition density $p_t(x,y) = p_t(x-y)$ with characteristic function $$\E \left[ e^{i \langle \xi,X_t \rangle}\right] =  \int_\Rd e^{i \langle \xi,x \rangle}  \cdot p_t(x)\ dx = e^{-t||\xi||^\alpha}$$\\[12pt]
\end{enumerate}
\end{frame}

\begin{frame}{Introduction: $\alpha$-Stable L\'{e}vy Processes $X = (X_t)_{t\geq0}$}
\begin{figure}[h]
 \centering
\includegraphics[scale=0.35]{"Processes"}
\end{figure}
\end{frame}


\begin{frame}{$\alpha$-Stable L\'{e}vy Processes: Main Properties}
\begin{enumerate}
\item $\alpha = 2:$ Brownian motion\\[12pt]
\item Non-continuous stochastic process for $\alpha \in (0,2)$\\[12pt]
\item $X$ is self-similar, i.e. $X_t \ \eqfd \ t^{1/\alpha}  \cdot X_1 $ \\[12pt]
\item Models non-local phenomena, i.e. jumps are involved\\[12pt]
\item Translates fractional heat equation to particle scale
\end{enumerate}
\end{frame}


\begin{frame}{Translation of PDEs to Stochastic Processes}
\begin{gather*}
\text{The fractional heat equation}\\[12pt]
\dot{v} + (-\Delta)^{\alpha/2}[v] = 0, \ v|_{t=0} = v_0\\[12pt]
\text{is solved by}\\[12pt]
e^{-t \, (-\Delta)^{\alpha/2}}[v_0](x) = \int_\Rd p(t,x,y) \cdot v_0(y)\ dy = \E[v_0(x + X_t)]
\end{gather*}
\end{frame}

\begin{frame}{Stable L\'{e}vy Processes are useful}
\begin{enumerate}
\item Model non-local particle movement by including jumps\\[12pt]
\item Subsumes Brownian motion and jump processes\\[12pt]
\item (Fractional) Brownian motion is self-similar \textbf{and} continuous\\[12pt]
\item All other stable processes are self-similar,  \textbf{not} continuous\\[12pt]
\item \textbf{Stable processes are testfunctions for self-similarity!}
\end{enumerate}
\end{frame}


\begin{frame}{Stochastic Formulation of PDEs}
\begin{gather*}
\text{PDE with initial value } v|_{t=0}=v_0\\
\quad\\
\text{Rule for } v_t \ \text{without time derivatives}\\
\quad\\
\text{SDE with initial value } X_0(x)=x
\end{gather*}
\end{frame}

\begin{frame}{Inviscid Burgers Equation}
\begin{gather*}
\dot{v}+(v\cdot \nabla)v \color{lightgray} - \mu \Delta v + \nabla p \color{black} = 0 \color{lightgray},  \quad \text{div} \ v = 0 \\
\quad\\
v_t(y) = \color{lightgray} \mathbb{E} \textbf{P} [ \nabla Y_t^{-1}(y) \cdot \color{black} v_0(Y_t^{-1}(y))\color{lightgray}]\\
\quad \\
Y_t(x) = x + \int_0^t v_s(x) \, ds  \color{lightgray} + \sqrt{2\mu} B_t \color{black} = y
\end{gather*}
\end{frame}


\begin{frame}{Inviscid Burgers Equation}
\begin{figure}[h]
 \centering
\includegraphics[scale=0.55]{"Burgers_equation"}
\end{figure}
\end{frame}


\begin{frame}{Viscid Burgers Equation}
\begin{gather*}
\dot{v}+(v\cdot \nabla)v - \mu \Delta v \color{lightgray} + \nabla p \color{black} = 0 \color{lightgray} ,  \quad \text{div} \ v = 0 \\
\quad\\
v_t(y) = \mathbb{E} \color{lightgray} \textbf{P} \color{black}[ \color{lightgray} \nabla Y_t^{-1}(y) \cdot \color{black} v_0(Y_t^{-1}(y))]\\
\quad \\
Y_t(x) = x + \int_0^t v_s(x) \, ds  + \sqrt{2\mu} B_t \color{black} = y
\end{gather*}
\end{frame}


\begin{frame}{Inviscid Burgers Equation}
\begin{gather*}
\dot{v}+(v\cdot \nabla)v \color{lightgray} - \mu \Delta v + \nabla p \color{black} = 0 \color{lightgray},  \quad \text{div} \ v = 0 \\
\quad\\
v_t(y) = \color{lightgray} \mathbb{E}  \textbf{P} [ \nabla Y_t^{-1}(y) \cdot \color{black} v_0(Y_t^{-1}(y))\color{lightgray}]\\
\quad \\
Y_t(x) = x + \int_0^t v_s(x) \, ds  \color{lightgray} + \sqrt{2\mu} B_t \color{black} = y
\end{gather*}
\end{frame}

\begin{frame}{Euler Equations}
\begin{gather*}
\dot{v}+(v\cdot \nabla)v \color{lightgray} - \mu \Delta v \color{black} + \nabla p = 0,  \quad \text{div} \ v = 0\\
\quad\\
v_t(y) = \color{lightgray} \mathbb{E} \color{black} \textbf{P}[ \nabla Y_t^{-1}(y) \cdot v_0(Y_t^{-1}(y))]\\
\quad \\
Y_t(x) = x + \int_0^t v_s(x) \, ds  \color{lightgray} + \sqrt{2\mu} B_t \color{black} = y
\end{gather*}
\end{frame}



\begin{frame}{Navier-Stokes Equations}
\begin{gather*}
\dot{v}+(v\cdot \nabla)v - \mu \Delta v + \nabla p = 0,  \quad \text{div} \ v = 0 \\
\quad\\
v_t(y) = \mathbb{E} \textbf{P}[ \nabla Y_t^{-1}(y) \cdot v_0(Y_t^{-1}(y))]\\
\quad \\
Y_t(x) = x + \int_0^t v_s(x) \, ds  + \sqrt{2\mu} B_t \color{black} = y
\end{gather*}
\end{frame}


\begin{frame}{Stochastic Process plus Deterministic Driftfunction}
\begin{gather*}
\dot{v}+(v\cdot \nabla)v - \mu \Delta v + \nabla p = 0,  \quad \text{div} \ v = 0 \\
\quad\\
v_t(y) = \mathbb{E} \textbf{P}[ \nabla Y_t^{-1}(y) \cdot v_0(Y_t^{-1}(y))]\\
\quad \\
Y_t(x) = \color{offred}{x + \int_0^t v_s(x) \, ds  + \sqrt{2\mu} B_t} \color{black} = y
\end{gather*}
\end{frame}


\begin{frame}{Parabolic Hausdorff Measure and Dimension}

\begin{center}
Restricted Hausdorff measure:
\end{center}

\begin{gather*}
\mathcal{P}^\alpha\m \mathcal{H}^\beta(A) := \lim_{\delta \downarrow 0}\ \inf \bigg\{\sum_{n=1}^\infty |\mathsf{P}_n|^\beta: A\subseteq\bigcup_{n=1}^\infty \mathsf{P}_n,\ \mathsf{P}_n \in \mathcal{P}^\alpha,\ |\mathsf{P}_n|\leq \delta \bigg\} \\
\quad\\
\mathcal{P}^\alpha\m\dim A := \sup\, \{\beta : \mathcal{P}^\alpha\m\pazocal{H}^\beta(A) = \infty \big\} =  \inf \big\{\beta : \mathcal{P}^\alpha\m\pazocal{H}^\beta(A) =0\}
\end{gather*}

\end{frame}

\begin{frame}{Parabolic Cylinders}
\begin{center}
Distinct non-linear scaling between time and space:
\end{center}

\begin{gather*}
[Taylor, Watson, 1985]\\
\mathcal{P}^2=\bigg \{ [t,t+r^2]\times \prod_{i=1}^d\ [x_{i},x_{i}+r],\ c \in (0,1] \bigg\}\\
\quad \\
[Peres, Sousi, 2016]\\
\mathcal{P}^\alpha = \bigg \{ [t,t+c]\times \prod_{i=1}^d\ [x_{i},x_{i}+c^{1/\alpha}],\ c \in (0,1] \bigg\}
\end{gather*}

\end{frame}


\begin{frame}{Formulas: Hausdorff Dimension for $X+f$ }
Let f be a Borel measurable function, $\ph_\alpha := \mathcal{P}^\alpha\m\dim \pazocal{G}_T(f)$ where $\ph_1 = \dim \pazocal{G}_T(f)$.  Then a.s.
\begin{equation*}
\dim \pazocal{G}_T(X+f) =
\begin{cases}
\ph_1, & \alpha \in (0,1],\\
\ph_\alpha \ \wedge \ \frac{1}{\alpha} \cdot \ph_\alpha + \big(1-\frac{1}{\alpha}\big)\cdot d, & \alpha \in [1,2].
\end{cases}
\end{equation*}
and
\begin{equation*}
\dim \pazocal{R}_T(X+f) =
\begin{cases}
\alpha \cdot \ph_\alpha \ \wedge \ d, & \alpha \in (0,1],\\
\ph_\alpha \ \wedge \ d, & \alpha \in [1,2].
\end{cases}
\end{equation*}
\end{frame}

\begin{frame}{Upper Bounds: Geometric Measure Theory}
One a.s. has
\begin{equation*}
\mathcal{P}^\alpha\m\dim \pazocal{G}_T(X+f) \leq \mathcal{P}^\alpha\m\dim \pazocal{G}_T(f).
\end{equation*}
\begin{proof}[Proof idea:] Let $ \beta := \ph_\alpha$, let $M_k(\omega)$ be the random number of hypercubes with sidelength $c_k^{1/\alpha}$ that the path $t\mapsto X_t(\omega)$ hits. 
\begin{align*}
& \E\Big[\mathcal{P}^\alpha\m \mathcal{H}^{\beta+\eps}( \mathcal{G}_T(X+f))\Big]
\leq \E\Bigg[\sum_{k=1}^\infty |\mathsf{P}^\omega|^{\beta+\eps} \Bigg]\\
& \quad \lesssim \sum_{k=1}^\infty \E[M_k(\omega)]\cdot c_k^{\beta + \eps} \lesssim  \sum_{k=1}^\infty  c_k^{\beta + \delta} < \infty.
\end{align*}
Hence $\mathcal{P}^\alpha\m\dim \pazocal{G}_T(X+f) \leq \mathcal{P}^\alpha\m\dim \pazocal{G}_T(f)$, a.s.
\end{proof}
\end{frame}

\begin{frame}{Lower Bounds: Potential Theory}
Let $\ph_\alpha := \mathcal{P}^\alpha\m\dim \pazocal{G}_T(f)$ where $\ph_1 = \dim \pazocal{G}_T(f)$. Then a.s.
\begin{equation*}
\dim \pazocal{G}_T(X+f) \geq
\begin{cases}
\ph_1, & \alpha \in (0,1],\\
\ph_\alpha \ \wedge \ \frac{1}{\alpha} \cdot \ph_\alpha + \big(1-\frac{1}{\alpha}\big)\cdot d, & \alpha \in [1,2].
\end{cases}
\end{equation*}
\begin{proof}[Proof idea:] Let $K^\beta(t,x) =\E \big[||(t,  X_t(\omega) + x)||^{-\beta}\big]$. Then one a.s. has
\begin{align*}
& \pazocal{E}_{K^\beta}(\mu) = \int \int_{\pazocal{G}_T(f)\times \pazocal{G}_T(f)}  K^\beta(t-s,f(t)-f(s))\, \textnormal{d}\mu (s,x)\, \textnormal{d}\mu (t,y) < \infty
\end{align*}
for some probability measure $\mu \in \pazocal{M}^1(\pazocal{G}_T(f))$.
\end{proof}
\end{frame}

\begin{frame}{Parabolic Version of Frostman's Lemma}
Let $A \in \mathcal{B}(\R^{1 + d})$. If $\mathcal{P}^{\alpha}\m\dim A > \beta,$ then there exists $\mu \in \pazocal{M}^1(A)$ such that
\begin{equation*}
\mu\Bigg(\big[t,\, t+c\big]\times \prod_{i=1}^d \Big[x_i,\, x_i+c^{1/\alpha}\Big]\Bigg) \lesssim 
\begin{cases}
c^{\, \beta}, & \alpha \in (0,1],\\
c^{\, \beta/ \alpha}, & \alpha \in [1,\infty)
\end{cases}
\end{equation*} 
for every $c \in (0,1]$ and $t,  x_1,\dots,x_d \in \R$.
\end{frame}

\begin{frame}{Equalities for the Parabolic Hausdorff Dimension}
Define $\ph_f := \mathcal{P}^\alpha\m\dim \pazocal{G}_T(f)$. If $f \equiv C \in \Rd$, then a.s.
$$\ph_f = (\alpha \vee 1) \cdot \dim T. $$
If $f \equiv X \in \Rd$, then a.s.
$$\ph_X = (\alpha \vee 1) \cdot \dim T. $$
\end{frame}


\begin{frame}{Formulas: Hausdorff Dimension for $X$}

Since $X = X+0^d$ one has a.s.
\begin{equation*}
\dim \pazocal{G}_T(X) =
\begin{cases}
\dim T + 1 - 1/\alpha,  & d=1, \ \alpha\cdot \dim T > 1,\\
(\alpha \vee 1) \cdot \dim T,  & d=1, \ \alpha\cdot \dim T \leq 1\text{ or } d\geq2,
\end{cases} 
\end{equation*}
which was proved by \textit{Blumenthal} and \textit{Getoor} (1960).\\[30pt]

It fits the result from \textit{Taylor} (1953) for Brownian motion $B = (B_t)_{t\geq0}$:
\begin{equation*}
\dim \pazocal{G}_{[0,1]}(B) =
\begin{cases}
3/2  &\text{ for } d=1 ,\\
2,  &\text{ for } d\geq2.
\end{cases} 
\end{equation*}
\end{frame}

\begin{frame}{Estimates for the Parabolic Hausdorff Dimension}
Define $\ph_\alpha := \mathcal{P}^\alpha\m\dim \pazocal{G}_T(f)$ where $\ph_1 = \dim \pazocal{G}_T(f)$. Then a.s.
\begin{equation*}
\ph_\alpha \leq
\begin{cases}
\ph_1 + \big(\frac{1}{\alpha} - 1\big)\cdot d \ \wedge \ d + 1, & \alpha \in (0,1],\\
\ph_1 + \alpha - 1  \ \wedge \ d + 1, & \alpha \in [1,\infty)
\end{cases}
\end{equation*}
and
\begin{equation*}
\ph_\alpha \geq
\begin{cases}
\ph_1 \ \vee \ \frac{1}{\alpha} \cdot \ph_1 + 1 - \frac{1}{\alpha}, & \alpha \in (0,1],\\
\ph_1 \ \vee \ \alpha \cdot \ph_1 + (1-\alpha) \cdot d, & \alpha \in [1,\infty).
\end{cases}
\end{equation*}
\end{frame}

\begin{frame}{Estimates for the Parabolic Hausdorff Dimension}
Define $\ph_\alpha := \mathcal{P}^\alpha\m\dim \pazocal{G}_T(f)$, $\beta \in (0,1]$ and $f\in C^\beta\big(T,\Rd\big)$. Then a.s.
\begin{align*}
\ph_\alpha \leq
\begin{cases}
\dim T + d \cdot \big(\frac{1}{\alpha} - \beta \big) \ \wedge \ \frac{\dim T}{\alpha \beta} \ \wedge \ d+1, & \alpha \in (0,1],\\
\alpha \cdot \dim T + d \cdot (1 - \alpha\beta) \ \wedge \ \frac{\dim T}{\beta} \ \wedge \ d+1, & \alpha \in \big[1,\frac{1}{\beta}\big],\\
\alpha \cdot \dim T \ \wedge \ \frac{1}{\beta} \cdot (\dim T - 1) + \alpha \ \wedge \ d+1, & \alpha \in \big[\frac{1}{\beta},\infty\big)
\end{cases}
\end{align*}
\end{frame}

\begin{frame}{Conclusion}
\begin{enumerate}
\item Parabolic fractal geometry lets the stable process disappear.\\[20pt]
\item Difficult in lower dimension and easier in higher dimensions.\\[20pt]
\item Aim: Make use of the results for PDEs via polar sets.
\end{enumerate}
\end{frame}


\begin{frame}{Literature}

\quad\\[15pt]

\begin{enumerate}
\item \textbf{Peres},  \textbf{Sousi } (2016)

\textit{Dimension of Fractional Brownian motion with variable drift.}

Probab.  Theory Related Fields {\bf 165}, 771--94.\\

\quad\\

\item \textbf{Pruitt}, \textbf{Taylor} (1969)

\textit{Sample Path Properties of Processes with Stable Components.}

Z. Wahrscheinlichkeitstheorie verw. Gebiete {\bf 12}, 267--289.\\

\quad

\item \textbf{Taylor},  \textbf{Watson} (1985)

\textit{A Hausdorff measure classification of polar sets for the heat equation.}

Math. Proc. Cambridge Phil. Soc. {\bf 97}, 325--44.

\end{enumerate}

\quad\\[15pt]


\begin{center}
leonard.pleschberger@gmail.com\\[12pt]
\end{center}
\end{frame}

\end{document}

